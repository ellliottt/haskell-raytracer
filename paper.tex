\documentclass{article}

\usepackage{amsmath}
% \usepackage{minted}

\title{Realtime Ray Tracing}
\author{...}
\date{...}

\begin{document}

\maketitle

\tableofcontents

\section{Introduction}

    Raytracing is a rendering technique that can produce photorealistic results. It works by
    simulating the paths of light as it moves through the scene. Each light bounce is simulated by
    sending subsequent rays from the intersection point.

\section{Raytracing Algorithm}

    At first, achieving photorealistic rendering appears to requires simulating every photon in a
    scene. Without this simulation, the resulting image can only be a rough approximation of the
    complicated interactions occuring in the real world.

    A photon travels from the light source until it hits an object. Then, depending on the surface
    normal and material, one or more photons are emitted, absorbing some of the color of the
    material. This process is recursive, as each emitted photon must be traced. It is also
    exponential. If a given surface emits more than one photon, and assuming that each photon is
    guaranteed to hit another object, the number of photons in the scene will be given by $n^k$ for
    the initial number of photons $n$ and a maximum bouce threshold $k$. This means the number of
    simulated rays grows at an exponential rate with respect to the complexity of the scene. Such
    an algorithm is not scalable due to this factor. If a photon hits the camera, its colour is
    plotted as a pixel onto the output image. However, in most scenes, most photons will not hit
    the camera. They may continue bouncing until they hit the maximum number of bouces $k$, or they
    may not hit anything and continue moving away from the scene forever. This makes simulating
    each photon computationally infeasable. A very large number of photons must be simulated, and
    most of this computation will be useless in producing an image.

    A more practical, but still computationally difficult, method is to reverse the direction of
    simulation. Instead of tracing the path of each photon from the light source, rays can be sent
    out from the camera. The same check is performed for intersection for shapes, and recursive
    ray are emitted. A ray stops once it hits the maxiumum bounce threshold $k$, or hits a light
    source. While this seems counter-intuitive, as photons do not travel from the camera, by doing
    this, only the rays necessary for computing the output image are simulated.

    \subsection{Camera}
        The first step is to define a camera. It can be assumed that the camera point is centered at
        the origin and, by convention, pointing in the negative $z$ direction. The \emph{viewport} is a
        plane representing the screen, positioned perpendicular to the $z$ axis, intersecting the $z$
        axis in the plane's center, shifted along the $z$ axis by $-k$, where $k$ is the
        \emph{focal length} of the camera. Thus, the equation for a viewport of size $n\times m$ is
        given by:
        \begin{equation*}
            z = -k, -\frac{n}{2} \leq x \leq \frac{n}{2}, -\frac{m}{2} \leq y \leq \frac{m}{2}
        \end{equation*}
        Rays can be sent from the origin through the position on the viewport that each pixel would be,
        and the colour of that ray can be plotted at that pixel location.

    \subsection{Primary Ray}
        Next is computing the \emph{primary ray}. This is the initial ray that is sent by the camera
        into the scene. A ray is nothing more than a line in 3 dimensions, so the primary ray is given
        by:
        \begin{equation*}
            ux + vy - kz = 0, -\frac{n}{2} \leq u \leq \frac{n}{2}, -\frac{m}{2} \leq v \leq \frac{m}{2}
        \end{equation*}
        where $u$ and $v$ control the position of the intersection with the viewport, representing the
        pixel to be calculated. This ray then extends through the viewport and into the scene.

    \subsection{Scene Intersection}
        The closest intersection between the primary ray and the scene can be found by taking the
        minimum of $\delta S$, for each shape $S$, and a distance function $\delta$. This distance
        function can be defined for each primitive shape, such as spheres, cubes, and planes, and
        can be written for triangles to allow for arbitrary mesh rendering.

        \subsubsection{Sphere Intersection}
            The equation of a sphere of radius $r$ is given by:
            \begin{equation*}
                x^2 + y^2 + z^2 = r^2
            \end{equation*}
            This means that any point $P$ that satisfies the above equation must be on the surface
            of the sphere. For an arbitrary center $C$, the equation is:
            \begin{equation*}
                (x - C_x)^2 + (y - C_y)^2 + (z - C_z)^2 = r^2
            \end{equation*}
            As $P$ and $C$ are vectors, this can be written as:
            \begin{equation*}
                (P - C) \cdot (P - C) - r^2 = 0
            \end{equation*}

        \subsubsection{Plane Intersection}



\end{document}

